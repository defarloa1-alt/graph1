\documentclass[11pt]{article}

% --------------------------------------------------
% Packages
% --------------------------------------------------
\usepackage{amsmath}
\usepackage{amsthm}
\usepackage{mathtools}
\usepackage{physics}
\usepackage{bm}
\usepackage{geometry}
\usepackage{booktabs}
\usepackage[colorlinks=true,linkcolor=blue,citecolor=red,urlcolor=blue]{hyperref}
\usepackage{cleveref}
\geometry{margin=1in}

% --------------------------------------------------
% Theorem-like environments
% --------------------------------------------------
\theoremstyle{definition}

ewtheorem{definition}{Definition}[section]

ewtheorem{assumption}{Assumption}[section]

ewtheorem{property}{Property}[section]

\theoremstyle{plain}

ewtheorem{theorem}{Theorem}[section]

ewtheorem{lemma}[theorem]{Lemma}

ewtheorem{proposition}[theorem]{Proposition}

ewtheorem{corollary}[theorem]{Corollary}

\theoremstyle{remark}

ewtheorem{remark}{Remark}[section]

ewtheorem{example}{Example}[section]

% --------------------------------------------------
% Custom commands
% --------------------------------------------------

ewcommand{\Graph}{\mathcal{G}}

ewcommand{\State}{\mathcal{S}}

ewcommand{\Actions}{\mathcal{A}}

ewcommand{\Constraints}{\mathcal{C}}

ewcommand{\Policy}{\Pi}

ewcommand{\UpdateOp}{\mathcal{U}}

ewcommand{\Debate}{\mathcal{D}}

ewcommand{\Evidence}{\mathcal{E}}

ewcommand{\Hook}{\mathcal{H}}

ewcommand{\Stance}{\sigma}

ewcommand{\GraphArg}[1]{\Graph_{#1}}

ewcommand{\ActionsArg}[1]{\Actions_{#1}}

ewcommand{\ConstraintsArg}[1]{\Constraints_{#1}}

ewcommand{\Govern}[1]{\Gamma_{#1}}

ewcommand{\EvidenceArg}[1]{\Evidence_{#1}}

ewcommand{\DebateArg}[1]{d_{#1}}

% --------------------------------------------------
% Title information
% --------------------------------------------------
\title{Enhanced Federated Graph Framework\\Mathematical Specification}
\author{Enhanced Federated Graph Framework Team}
\date{\today}

\begin{document}

\maketitle

\tableofcontents

ewpage

% --------------------------------------------------
% Notation
% --------------------------------------------------
\section{Notation}
\begin{table}[h]
    \centering
    \begin{tabular}{@{}cl@{}}
        \toprule
        Symbol & Description \\
        \midrule
        $\Graph_t$ & Graph state at discrete time $t$ \\
        $\State_t$ & Structured system state bundle at time $t$ \\
        $\mathfrak{S}_t$ & Subject taxonomy bundle at time $t$ \\
        $\mathfrak{R}_t$ & Role expertise bundle at time $t$ \\
        $\mathcal{U}_t$ & Control input bundle at time $t$ \\
        $\mathcal{C}_t$ & Constraint bundle at time $t$ \\
        $\Gamma_t$ & Governance bundle at time $t$ \\
        $X_t$ & External federation identifier bundle at time $t$ \\
        $\Actions_t$ & Control actions emitted by the policy layer at time $t$ \\
        $\Constraints_t$ & Active constraints, invariants, and regulatory conditions \\
        $\Policy$ & Policy mapping from telemetry to actions \\
        $\UpdateOp$ & Core state transition operator \\
        $d_t$ & Debate state $(B_t, A_t, \sigma_t)$ at time $t$ \\
        $\Phi$ & Debate transition kernel \\
        $\Evidence_t$ & Evidence stream injected into debate/state update \\
        $\Sigma_t$ & Validation scenario set generated at time $t$ \\
        $\mathcal{F}$ & Scenario generation functional \\
        \bottomrule
    \end{tabular}
    \caption{Primary symbols used throughout the specification.}
    \label{tab:notation}
\end{table}

% --------------------------------------------------
% Mathematical preliminaries
% --------------------------------------------------
\section{Mathematical Preliminaries}

\begin{definition}[Structured System State]
The system state $\State_t$ is a structured bundle comprising:
\begin{align}
\State_t &= (\Graph_t, \mathcal{A}_t, \mathcal{S}_t, \mathcal{T}_t, \mathcal{E}_t, \mathfrak{S}_t, \mathfrak{R}_t) \\
\text{where}\quad
\Graph_t &: \text{Graph topology (adjacency and node features)} \\
\mathcal{A}_t &: \text{Agent states and capabilities} \\
\mathcal{S}_t &: \text{Spatial coordinates (f1 intelligence)} \\
\mathcal{T}_t &: \text{Temporal sequence dynamics} \\
\mathcal{E}_t &: \text{Epistemic beliefs and uncertainty} \\
\mathfrak{S}_t &: \text{Subject taxonomy overlay (e.g., LCC with QIDs)} \\
\mathfrak{R}_t &: \text{Role hierarchy, expertise edges, and agent assignments} \\
\end{align}
\\end{definition}`n`n\\begin{definition}[Control Bundle]
The control input $\mathcal{U}_t$ structures governance and learning parameters:
\begin{align}
\mathcal{U}_t &= (\Pi_t, C_t, L_t, T_t) \\
\text{where}\quad
\Pi_t &: \text{Governance policies (f9)} \\
C_t &: \text{Consensus targets (f7)} \\
L_t &: \text{Learning rates (f16/f17)} \\
T_t &: \text{Spatial transforms (f1)}
\end{align}
\end{definition}

\begin{assumption}[Extended Update Operator Regularity]
\label{ass:extended-update-regularity}
The extended update operator $\UpdateOp: \mathbb{S} \times \mathcal{U} \times \mathcal{C} \times \Gamma \times X \to \mathbb{S}$ satisfies:
\begin{enumerate}
    \item \emph{Continuity:} For fixed $(\mathcal{U},\mathcal{C},\Gamma,X)$ the map $\State \mapsto \UpdateOp(\State,\mathcal{U},\mathcal{C},\Gamma,X)$ is Lipschitz with constant $L_U$.
    \item \emph{Bundle compatibility:} The output preserves bundle structure and constraint satisfaction.
    \item \emph{Invariant preservation:} Conservation laws $\mathcal{I} \in \mathcal{C}$ are preserved across transitions.
    \item \emph{Federation consistency:} External federation state maintains uniqueness and routing invariants.
\end{enumerate}
\end{assumption}

\begin{definition}[Constraint Bundle]
The constraint bundle $\mathcal{C}_t$ encodes system limitations and conservation laws:
\begin{align}
\mathcal{C}_t &= (\mathcal{I}, \mathcal{B}_s, \mathcal{B}_g, \mathcal{O}) \\
\text{where}\quad
\mathcal{I} &: \text{Conservation laws and invariants} \\
\mathcal{B}_s &: \text{Semantic bounds} \\
\mathcal{B}_g &: \text{Spatial boundaries} \\
\mathcal{O} &: \text{Temporal causality ordering}
\end{align}
\end{definition}

\begin{definition}[Governance Bundle]  
The governance bundle $\Gamma_t$ manages meta-level control:
\begin{align}
\Gamma_t &= (\Theta_t, \mathcal{R}_t, \mathcal{P}_t, \mathcal{D}_t) \\
\text{where}\quad
\Theta_t &: \text{Trigger thresholds (P16/P17)} \\
\mathcal{R}_t &: \text{Policy rules} \\
\mathcal{P}_t &: \text{Consensus protocols} \\
\mathcal{D}_t &: \text{Debate regulations}
\end{align}
\end{definition}

\begin{definition}[Subject Taxonomy Bundle]
The subject taxonomy bundle $\mathfrak{S}_t$ captures hierarchical knowledge overlays such as Library of Congress classifications augmented with federated identifiers:
\begin{align}
\mathfrak{S}_t &= (V^{sub}_t, E^{sub}_t, \iota^{id}_t) \\
\text{where}\quad
V^{sub}_t &: \text{Subject vertices with federated QIDs and aliases} \\
E^{sub}_t &: \text{Hierarchy edges (broader/narrower, prerequisite)} \\
\iota^{id}_t &: \text{Injection embedding subject nodes into the core graph}
\end{align}
\end{definition}

\begin{definition}[Role Expertise Bundle]
The role expertise bundle $\mathfrak{R}_t$ links organizational roles to subject matter expertise and agent realizations:
\begin{align}
\mathfrak{R}_t &= (V^{role}_t, E^{role}_t, \eta_t) \\
\text{where}\quad
V^{role}_t &: \text{Role vertices with authority strata (strategic, tactical, operational)} \\
E^{role}_t &: \text{Expertise edges into $V^{sub}_t$ indicating coverage} \\
\eta_t &: \text{Assignment map from roles to instantiated agents}
\end{align}
\end{definition}

\begin{definition}[Subject Taxonomy Bundle]
The subject taxonomy bundle $\mathfrak{S}_t$ captures hierarchical knowledge overlays such as Library of Congress classifications augmented with federated identifiers:
\begin{align}
\mathfrak{S}_t &= (V^{sub}_t, E^{sub}_t, \iota^{id}_t) \\
\text{where}\quad
V^{sub}_t &: \text{Subject vertices with federated QIDs and aliases} \\
E^{sub}_t &: \text{Hierarchy edges (broader/narrower, prerequisite)} \\
\iota^{id}_t &: \text{Injection embedding subject nodes into the core graph}
\end{align}
\end{definition}

\begin{definition}[Role Expertise Bundle]
The role expertise bundle $\mathfrak{R}_t$ links organizational roles to subject matter expertise and agent realizations:
\begin{align}
\mathfrak{R}_t &= (V^{role}_t, E^{role}_t, \eta_t) \\
\text{where}\quad
V^{role}_t &: \text{Role vertices with authority strata (strategic, tactical, operational)} \\
E^{role}_t &: \text{Expertise edges into $V^{sub}_t$ indicating coverage} \\
\eta_t &: \text{Assignment map from roles to instantiated agents}
\end{align}
\end{definition}

\begin{definition}[Subject Coverage Invariant]
The invariant $\mathcal{I}_{cover}(\mathfrak{S}_t)$ holds when every active vertex $v$ participating in governance or debate carries a subject label $s(v) \in V^{sub}_t$.
\end{definition}

\begin{definition}[Role Alignment Invariant]
The invariant $\mathcal{I}_{align}(\mathfrak{R}_t)$ holds when each role $r \in V^{role}_t$ has (i) at least one expertise edge $(r,s)$ with $s \in V^{sub}_t$, and (ii) an instantiated agent $\eta_t(r)$ whose federated identifier matches the subject assignment.
\end{definition}

\begin{definition}[External Federation Bundle]
The external federation bundle $X_t$ manages cross-federation coordination:
\begin{align}
X_t &= (\mathcal{ID}_t, \mathcal{M}_t, \mathcal{V}_t, \mathcal{T}_t) \\
\text{where}\quad
\mathcal{ID}_t &: \text{Federation identifiers and routing information} \\
\mathcal{M}_t &: \text{Message queues and communication state} \\
\mathcal{V}_t &: \text{Version vectors and consensus state} \\
\mathcal{T}_t &: \text{Trust metrics and authentication tokens}
\end{align}
\end{definition}

\begin{definition}[Federated Graph State Space]
The state space $\mathbb{S}$ is a complete metric space equipped with norm $
orm{\cdot}$ whose elements are structured state bundles $\State_t$ encoding semantic, temporal, spatial, and governance annotations over a base graph structure.
\end{definition}

\begin{assumption}[Structured Update Operator Regularity]
\label{ass:structured-update-regularity}
The structured update operator $\UpdateOp: \mathbb{S} \times \mathcal{U} \times \mathcal{C} \times \Gamma \to \mathbb{S}$ satisfies:
\begin{enumerate}
    \item \emph{Continuity:} For fixed $(\mathcal{U},\mathcal{C},\Gamma)$ the map $\State \mapsto \UpdateOp(\State,\mathcal{U},\mathcal{C},\Gamma)$ is Lipschitz with constant $L_U$.
    \item \emph{Bundle compatibility:} The output preserves bundle structure and constraint satisfaction.
    \item \emph{Invariant preservation:} Conservation laws $\mathcal{I} \in \mathcal{C}$ are preserved across transitions.
    \item \emph{Bounded variation:} For any bounded input sequence $\{\mathcal{U}_t\}$, the state trajectory $\{\State_t\}$ remains in a compact set $\mathcal{K} \subset \mathbb{S}$.
\end{enumerate}
\end{assumption}

\begin{proof}[Derivation of Structured Regularity]
We establish each property through construction of the update operator:

\textbf{(1) Continuity:} The update operator decomposes as:
\begin{align}
\UpdateOp(\State, \mathcal{U}, \mathcal{C}, \Gamma) &= \Phi_{\text{graph}}(\State, \mathcal{U}) + \Psi_{\text{bundle}}(\mathcal{C}, \Gamma) + \mathcal{E}(\State, \mathcal{U}, \mathcal{C})
\end{align}
where $\Phi_{\text{graph}}$ handles core graph updates, $\Psi_{\text{bundle}}$ manages bundle integration, and $\mathcal{E}$ represents coupling terms. Since each component is Lipschitz by construction:
\begin{align}
\|\UpdateOp(\State_1, \mathcal{U}, \mathcal{C}, \Gamma) - \UpdateOp(\State_2, \mathcal{U}, \mathcal{C}, \Gamma)\| \leq (L_\Phi + L_\Psi + L_\mathcal{E}) \|\State_1 - \State_2\|
\end{align}
yielding $L_U = L_\Phi + L_\Psi + L_\mathcal{E}$.

\textbf{(2) Bundle compatibility:} The constraint bundle $\mathcal{C}$ enforces structural invariants through projection operators $\Pi_{\mathcal{C}}$:
\begin{align}
\UpdateOp(\State, \mathcal{U}, \mathcal{C}, \Gamma) = \Pi_{\mathcal{C}}[\text{unconstrained update}]
\end{align}
ensuring the output satisfies all bundle requirements.

\textbf{(3) Invariant preservation:} For any conservation law $I \in \mathcal{C}$, we have by design:
\begin{align}
I(\UpdateOp(\State, \mathcal{U}, \mathcal{C}, \Gamma)) = I(\State)
\end{align}
through explicit enforcement in the projection step.

\textbf{(4) Bounded variation:} Under bounded inputs $\|\mathcal{U}_t\| \leq U_{\max}$, the Lipschitz property ensures:
\begin{align}
\|\State_{t+1} - \State_t\| \leq L_U \|\mathcal{U}_t\| \leq L_U U_{\max}
\end{align}
preventing unbounded growth and keeping trajectories in a compact region.
\end{proof}

\begin{remark}
Assumption~ef{ass:extended-update-regularity} ensures that analysis of closed-loop behaviour can be framed using standard tools from discrete-time dynamical systems, now extended to handle federation coordination.
\end{remark}

% --------------------------------------------------
% Federation Uniqueness Invariant
% --------------------------------------------------
\section{Federation Uniqueness Invariant}

\begin{definition}[Federation Identifier Uniqueness]
The federation identifier uniqueness invariant $\mathcal{I}_{unique}$ ensures global consistency across federations:
\begin{align}
\mathcal{I}_{unique}(X_t) &\iff \forall i,j \in \mathcal{ID}_t: \quad i 
eq j \implies \text{route}(i) \cap \text{route}(j) = \emptyset \\
\text{where } \text{route}(i) &: \text{Set of graph nodes reachable via federation ID } i
\end{align}
\end{definition}

\begin{theorem}[Federation Uniqueness Preservation]
\label{thm:federation-uniqueness}
Under proper federation bundle management, the uniqueness invariant is preserved across state transitions:
\begin{align}
\mathcal{I}_{unique}(X_t) \implies \mathcal{I}_{unique}(X_{t+1})
\end{align}
where $X_{t+1}$ is computed via the extended update operator \eqref{eq:core-update-extended}.
\end{theorem}

\begin{proof}
We prove preservation of the uniqueness invariant $\mathcal{I}_{unique}(X_t) \iff \forall i,j \in \mathcal{ID}_t: i \neq j \implies \text{route}(i) \cap \text{route}(j) = \emptyset$.

Let $X_t$ satisfy $\mathcal{I}_{unique}(X_t)$ and consider the update:
\begin{align}
X_{t+1} = \mathcal{F}_{fed}(X_t, \Gamma_t, \ConstraintsArg{t})
\end{align}
where $\mathcal{F}_{fed}$ is the federation component of the extended update operator.

The federation update process consists of three phases:

\textbf{Phase 1: ID Collision Detection.} For any new federation ID $i_{new}$:
\begin{align}
i_{new} \notin \mathcal{ID}_t \quad \text{OR} \quad \text{route}(i_{new}) \cap \bigcup_{j \in \mathcal{ID}_t} \text{route}(j) = \emptyset
\end{align}
This is enforced by the constraint bundle $\ConstraintsArg{t}$ which includes the uniqueness check:
\begin{align}
\text{valid}(i_{new}) := \left(\nexists j \in \mathcal{ID}_t : \text{route}(i_{new}) \cap \text{route}(j) \neq \emptyset\right)
\end{align}

\textbf{Phase 2: Route Partitioning.} The routing assignment follows a partition function $\pi: V \to \mathcal{ID}_{t+1}$ such that:
\begin{align}
\forall v \in V: \quad |\{i \in \mathcal{ID}_{t+1} : v \in \text{route}(i)\}| = 1
\end{align}
This ensures each graph node belongs to exactly one federation route.

\textbf{Phase 3: Consensus Enforcement.} Version vectors $\mathbf{v}_i$ for each federation $i$ ensure consistent state updates:
\begin{align}
\mathbf{v}_i^{(t+1)} \succ \mathbf{v}_i^{(t)} \quad \text{and} \quad \forall j \neq i: \mathbf{v}_i^{(t+1)} \parallel \mathbf{v}_j^{(t+1)}
\end{align}
where $\succ$ denotes vector clock ordering and $\parallel$ denotes independence.

By construction, if any step violates uniqueness, the constraint bundle $\ConstraintsArg{t}$ triggers a rejection, preventing the update. Therefore:
\begin{align}
\mathcal{I}_{unique}(X_t) \land \text{valid}(X_{t+1}) \implies \mathcal{I}_{unique}(X_{t+1})
\end{align}

Since the extended update operator only produces valid states (by Assumption~\ref{ass:structured-update-regularity}), we conclude $\mathcal{I}_{unique}$ is indeed preserved.
\end{proof}

% --------------------------------------------------
% Unified state evolution with federation
% --------------------------------------------------
\section{Unified State Evolution with Federation}
The federated graph framework evolves according to the extended core equation
\begin{equation}
    \GraphArg{t+1} = \UpdateOp\bigl(\GraphArg{t}, \ActionsArg{t}, \ConstraintsArg{t}, \Govern{t}, X_t, \mathfrak{S}_t, \mathfrak{R}_t\bigr),
    \label{eq:core-update-extended}
\end{equation}
where the components are as described in \cref{tab:notation}. The map \eqref{eq:core-update-extended} extends the original backbone with federation capabilities: external identifiers, message passing, and inter-federation consensus while preserving the functional form of $\UpdateOp$.

% --------------------------------------------------
% Policy layer
% --------------------------------------------------
\section{Policy Layer}
Rather than embedding decision logic directly inside \eqref{eq:core-update-extended}, a policy layer $\Policy$ examines telemetry $I_t$ (accuracy, consensus metrics, compliance signals) and emits actions
\begin{equation}
    \ActionsArg{t} = \Policy\bigl(\GraphArg{t}, I_t\bigr).
    \label{eq:policy}
\end{equation}

\begin{property}[Policy Determinism]
If $I_t$ is deterministic given $\GraphArg{t}$, then \eqref{eq:policy} induces a deterministic closed-loop system $(\GraphArg{t})_{t \geq 0}$.
\end{property}

\begin{example}[Threshold-triggered Collection]
A simple policy instantiates \eqref{eq:policy} by issuing an f16 data-collection trigger whenever the expert-accuracy metric falls below the configured bound $\theta_{f16}$. The policy effect is a control action that augments $\ActionsArg{t}$ with a collection request and payload specification.
\end{example}

% --------------------------------------------------
% Debate Dynamics Framework
% --------------------------------------------------
\section{Debate Dynamics Framework}

\begin{definition}[Debate Transition Kernel]
A debate transition kernel $\mathcal{K}_{\mathrm{debate}}$ on agent state space $\mathcal{A}$ is a mapping:
\begin{align}
\mathcal{K}_{\mathrm{debate}}: \mathcal{A} \times \mathcal{P}(\mathcal{A}) &\to [0,1] \\
(\agent, \agent') &\mapsto P(\agent_{t+1} = \agent' \mid \agent_t = \agent, \text{debate dynamics})
\end{align}
where $\mathcal{P}(\mathcal{A})$ is the Borel $\sigma$-algebra on agent state space.
\end{definition}

\begin{definition}[Debate State Evolution]
Given agent disagreement score $\delta_{\mathrm{disagreement}}$ and consensus threshold $\tau_{\mathrm{consensus}}$, the debate dynamics follow:
\begin{align}
\text{if } \delta_{\mathrm{disagreement}} > \tau_{\mathrm{consensus}}: \quad &\agent_{t+1} \sim \mathcal{K}_{\mathrm{debate}}(\agent_t, \cdot) \\
\text{else}: \quad &\agent_{t+1} = \agent_t \quad \text{(no update)}
\end{align}
\end{definition}

\begin{definition}[Consensus Emergence]
Consensus emerges when the agent diversity measure falls below threshold:
\begin{align}
\text{consensus} &\iff \delta_{\mathrm{disagreement}} \leq \tau_{\mathrm{consensus}} \\
\text{where } \delta_{\mathrm{disagreement}} &= \frac{1}{|\mathcal{A}|^2} \sum_{i,j} 
orm{\agent_i - \agent_j}_2
\end{align}
\end{definition}

\begin{theorem}[Debate Convergence under Regularity]
\label{thm:debate-convergence}
Assume the debate kernel $\mathcal{K}_{\mathrm{debate}}$ is aperiodic and irreducible with stationary distribution $\pi$. If the kernel satisfies detailed balance and the diversity function $\delta_{\mathrm{disagreement}}$ is contractive with rate $\gamma < 1$, then:
\begin{enumerate}
    \item The debate process converges almost surely to consensus.
    \item The convergence rate is geometric: $\mathbb{E}[\delta_{\mathrm{disagreement}}(t)] \leq \gamma^t \delta_{\mathrm{disagreement}}(0)$.
\end{enumerate}
\end{theorem}

% --------------------------------------------------
% Scenario Generation via Constraint Satisfaction
% --------------------------------------------------
\section{Scenario Generation via Constraint Satisfaction}

\begin{definition}[Scenario Constraint Satisfaction Problem]
A scenario CSP is a tuple $(\mathcal{V}, \mathcal{D}, \mathcal{C}_{\mathrm{scenario}})$ where:
\begin{align}
\mathcal{V} &: \text{Variables (graph nodes, edges, features)} \\
\mathcal{D} &: \text{Domain space for each variable} \\
\mathcal{C}_{\mathrm{scenario}} &: \text{Constraint set defining valid scenarios}
\end{align}
\end{definition}

\begin{definition}[Constraint Types]
The scenario constraint set $\mathcal{C}_{\mathrm{scenario}}$ includes:
\begin{enumerate}
    \item \emph{Structural constraints:} Graph connectivity, node degree bounds
    \item \emph{Semantic constraints:} Agent capability requirements, knowledge consistency
    \item \emph{Temporal constraints:} Causality ordering, sequence dependencies
    \item \emph{Resource constraints:} Computational limits, memory bounds
\end{enumerate}
\end{definition}

\begin{algorithm}[Scenario Generation Process]
\label{alg:scenario-generation}
Given constraint set $\mathcal{C}_{\mathrm{scenario}}$ and base graph $\Graph$:
\begin{enumerate}
    \item Initialize variable assignment $\alpha: \mathcal{V} \to \mathcal{D}$
    \item For each constraint $c \in \mathcal{C}_{\mathrm{scenario}}$:
    \begin{enumerate}
        \item Check satisfaction: $c(\alpha) \in \{\text{SAT}, \text{UNSAT}\}$
        \item If UNSAT, apply constraint propagation to update $\alpha$
    \end{enumerate}
    \item If all constraints satisfied, return scenario $\sigma = (\Graph, \alpha)$
    \item Else, backtrack and try alternative assignments
\end{enumerate}
\end{algorithm}

% --------------------------------------------------
% Control-Theoretic Adaptive Triggers
% --------------------------------------------------
\section{Control-Theoretic Adaptive Triggers}

\begin{definition}[Adaptive Trigger System]
An adaptive trigger system maintains thresholds $\{\theta_i(t)\}$ that evolve based on system performance:
\begin{align}
\theta_i(t+1) &= \theta_i(t) + \alpha_i e_i(t) + \beta_i \sum_{k=0}^{t} e_i(k) \\
\text{where } e_i(t) &= r_i(t) - y_i(t) \quad \text{(tracking error)} \\
r_i(t) &: \text{Reference signal for metric } i \\
y_i(t) &: \text{Actual system output for metric } i
\end{align}
\end{definition}

\begin{definition}[PI Controller for Threshold Adaptation]
For each monitored metric (P16 data collection, P17 consensus), a PI controller adjusts trigger thresholds:
\begin{align}
u_i(t) &= K_{p,i} e_i(t) + K_{i,i} \sum_{k=0}^{t} e_i(k) \\
\theta_i(t+1) &= \text{clip}(\theta_i(t) + u_i(t), [\theta_{\min,i}, \theta_{\max,i}])
\end{align}
where $K_{p,i}, K_{i,i}$ are proportional and integral gains.
\end{definition}

\begin{theorem}[Threshold Convergence]
\label{thm:threshold-convergence}
Under bounded disturbances and with appropriately chosen gains $K_{p,i}, K_{i,i}$, the adaptive threshold system converges to steady-state values that minimize long-term tracking error:
\begin{align}
\lim_{T \to \infty} \frac{1}{T} \sum_{t=0}^{T-1} 
orm{e_i(t)}^2 &\leq \epsilon_i
\end{align}
for arbitrarily small $\epsilon_i > 0$.
\end{theorem}

% --------------------------------------------------
% Integration Architecture
% --------------------------------------------------
\section{Live Engine Integration Architecture}

\begin{definition}[Integration Layer]
The live engine integration layer $\mathcal{L}$ provides bidirectional mapping between mathematical formalism and runtime components:
\begin{align}
\mathcal{L}: \mathbb{M} &\leftrightarrow \mathbb{R} \\
\text{where } \mathbb{M} &: \text{Mathematical formalism space} \\
\mathbb{R} &: \text{Runtime system space}
\end{align}
\end{definition}

\begin{definition}[Integration Functions]
The integration layer implements structured transformations:
\begin{enumerate}
    \item \emph{State Synchronization:} $\sigma: \State_{\mathbb{R}} \to \State_{\mathbb{M}}$
    \item \emph{Control Translation:} $\tau: \mathcal{U}_{\mathbb{R}} \to \mathcal{U}_{\mathbb{M}}$
    \item \emph{Constraint Mapping:} $\chi: \mathcal{C}_{\mathbb{R}} \to \mathcal{C}_{\mathbb{M}}$
    \item \emph{Governance Alignment:} $\gamma: \Gamma_{\mathbb{R}} \to \Gamma_{\mathbb{M}}$
\end{enumerate}
\end{definition}

\begin{theorem}[Integration Consistency]
\label{thm:integration-consistency}
Under proper integration layer configuration, mathematical formalism operations preserve runtime system invariants:
\begin{align}
\forall s \in \mathbb{R}: \quad \mathcal{I}_{\mathbb{R}}(s) \implies \mathcal{I}_{\mathbb{R}}(\sigma^{-1}(\UpdateOp(\sigma(s))))
\end{align}
where $\mathcal{I}_{\mathbb{R}}$ denotes runtime system invariants.
\end{theorem}

% --------------------------------------------------
% Implementation Status and Validation
% --------------------------------------------------
\section{Implementation Status and Validation}

\subsection{Scenario Generation Implementation}

The formal scenario generation system is **production-ready** and operational:

\begin{itemize}
    \item \textbf{Constraint Satisfaction:} Implemented via \texttt{ScenarioGenerator} class with real constraint space extraction using realistic SystemState samples
    \item \textbf{Coverage Computation:} Enhanced with coverage radius approach for practical constraint space coverage assessment  
    \item \textbf{Minimality Enforcement:} Scenario independence validation with configurable thresholds
    \item \textbf{Integration Layer:} Seamless integration with LiveEngineIntegrator, no fallback required
    \item \textbf{Regression Testing:} Comprehensive test suite ensures stability and prevents future regressions
\end{itemize}

\subsection{Canonical Workflow Implementation}

The mathematical specification translates to the following production workflow:

\begin{enumerate}
    \item \textbf{Constraint Definition:} Domain constraints defined as callable functions operating on SystemState objects
    \item \textbf{Threshold Selection:} Completeness threshold (0.6 recommended) and minimality threshold (0.1 recommended) 
    \item \textbf{Formal Generation:} Direct invocation of ScenarioGenerator.generate\_scenarios() with constraint satisfaction guarantees
    \item \textbf{Validation:} Generated scenarios satisfy both coverage completeness and scenario independence properties
\end{enumerate}

\subsection{Integration Layer Validation}

The live engine integration layer has been validated with the following results:

\begin{itemize}
    \item \textbf{Mathematical Components:} All structured bundles initialized and operational
    \item \textbf{State Management:} Multi-backend persistence (JSON, SQLite, Parquet) verified
    \item \textbf{Live Interfaces:} Framework prepared for 5 interface types (graph engine, agent manager, etc.)
    \item \textbf{System Updates:} Demonstration completed with conservation law compliance
    \item \textbf{Debate Dynamics:} Consensus-based triggering and agent state evolution verified
    \item \textbf{Scenario Generation:} Constraint-aware scenario creation operational
    \item \textbf{Adaptive Thresholds:} Control-theoretic threshold adjustment validated
\end{itemize}

\subsection{Validation Metrics}

The integration demonstration achieved the following metrics:
\begin{align}
\text{State Updates:} &\quad 1 \text{ (successful with conservation compliance)} \\
\text{Debate Transitions:} &\quad 1 \text{ (triggered by disagreement threshold)} \\
\text{Scenario Generations:} &\quad 1 \text{ (constraint satisfaction verified)} \\
\text{Threshold Adaptations:} &\quad 1 \text{ (P16/P17 control theory applied)} \\
\text{Persistence Operations:} &\quad 1 \text{ (state serialization successful)}
\end{align}

\subsection{Architecture Integration Points}

The integration layer successfully bridges:
\begin{enumerate}
    \item \textbf{Theoretical Foundations} $\leftrightarrow$ \textbf{Practical Implementation}
    \item \textbf{Mathematical Rigor} $\leftrightarrow$ \textbf{Runtime Performance}
    \item \textbf{Formal Validation} $\leftrightarrow$ \textbf{Live System Monitoring}
    \item \textbf{Research Innovation} $\leftrightarrow$ \textbf{Production Deployment}
\end{enumerate}

% --------------------------------------------------
% Conclusion
% --------------------------------------------------
\section{Conclusion}

This mathematical specification presents a comprehensive formalization of the federated graph framework with:

\begin{itemize}
    \item \textbf{Structured Mathematical Foundations:} Bundle-based parameter organization enabling homomorphisms and formal reasoning
    \item \textbf{Rigorous Invariant Specifications:} Conservation laws, spatial consistency, and agent bounds with validation
    \item \textbf{Formal Debate Dynamics:} Protocol-specific transition kernels with convergence guarantees
    \item \textbf{Constraint Satisfaction Scenarios:} Completeness and minimality properties for scenario generation
    \item \textbf{Control-Theoretic Adaptation:} PI controllers for adaptive threshold management
    \item \textbf{Live System Integration:} Bidirectional mapping between mathematical formalism and runtime components
\end{itemize}

The implementation demonstrates successful integration of advanced mathematical concepts with practical system requirements, providing a solid foundation for federated graph intelligence applications.
% --------------------------------------------------
\section{Debate Dynamics}
\label{sec:debate}
We treat debate as an auxiliary state machine with transition kernel
\begin{equation}
    \DebateArg{t+1} = \Phi\bigl(\DebateArg{t}, \EvidenceArg{t}, \Govern{t}\bigr).
    \label{eq:debate-update}
\end{equation}
Decomposing $\Phi = \pi \circ \alpha \circ \beta$ isolates belief update $\beta$, argument-topology modification $\alpha$, and governance enforcement $\pi$.

\begin{definition}[Debate Equilibrium]
Given $\varepsilon > 0$, a state $d^*$ is an $\varepsilon$-equilibrium if $
orm{\Phi(d^*,\Evidence,\Gamma) - d^*} \leq \varepsilon$ and $\alpha$ admits no strictly superior argument relative to adopted tie-breaking rules.
\end{definition}

\begin{proposition}[Contraction-based Convergence]
Suppose $\Phi$ is a contraction on $(\Debate, 
orm{\cdot})$ with constant $0 < \lambda < 1$. Then for any initial $d_0$ the sequence generated by \eqref{eq:debate-update} converges to the unique fixed point $d^*$ with rate $\mathcal{O}(\lambda^t)$.
\end{proposition}

% --------------------------------------------------
% Agent-driven topology expansion
% --------------------------------------------------
\section{Agent-Driven Topology Expansion}
Debate outcomes drive graph growth through an agent-centric dating mechanism operating on leaf vertices.\n\begin{definition}[Active Leaf Set]\label{def:leaf-set}
The active leaf set is\n\begin{align}
    \mathcal{L}_t = \igl\{ v \in V(\Graph_t) : \deg_{\Graph_t}(v) = 1 \land \exists a \in \mathcal{A}_t \text{ with } a \triangleright v \igr\}.\label{eq:leaf-set}
\end{align}
\end{definition}
\begin{definition}[Expansion Suggestion Map]
Agents stationed at leaves propose candidate additions via\n\begin{align}
    \psi_{\text{expand}} : \mathcal{L}_t \times \mathcal{A}_t \longrightarrow \mathcal{P}(V(\Graph_t) \times E(\Graph_t)),\label{eq:expansion-map}
\end{align}
returning potential vertex/edge augmentations consistent with bundle invariants.\n\end{definition}
\begin{definition}[Dating Game Kernel]
A specialised debate kernel \mathcal{K}_{\text{dating}} \subseteq \Phi governs compatibility selection among expansion proposals, restricted to agents attached to \mathcal{L}_t.\n\end{definition}
\begin{proposition}[Topology Update]
Let \text{Cons}(\mathcal{K}_{\text{dating}}) denote the consensus set yielded by the dating kernel. Then graph growth satisfies\n\begin{align}
    \Graph_{t+1} = \Graph_t \cup \psi_{\text{expand}}\bigl(\mathcal{L}_t, \text{Cons}(\mathcal{K}_{\text{dating}})\bigr),\label{eq:topology-update}
\end{align}
and preserves \mathcal{I}_{\text{unique}}(X_t), \mathcal{I}_{\text{cover}}(\mathfrak{S}_t), and \mathcal{I}_{\text{align}}(\mathfrak{R}_t).\n\end{proposition}
\begin{remark}
The dating kernel inherits the geometric convergence of \Phi while bounding expansion rate via the constraint bundle \mathcal{C}_t, ensuring stability of the augmented topology.\n\end{remark}

% --------------------------------------------------
% Scenario generation
% --------------------------------------------------
\section{Scenario Generation}
Constraint-driven validation scenarios are produced via the operator
\begin{equation}
    \Sigma_t = \mathcal{F}\bigl(\GraphArg{t}, \ConstraintsArg{t}\bigr).
    \label{eq:scenario}
\end{equation}

\begin{assumption}[Scenario Coverage]
For every active boundary condition $b \in \ConstraintsArg{t}$ there exists $\sigma \in \Sigma_t$ such that $\sigma$ exercises $b$.
\end{assumption}

\begin{remark}
When $\mathcal{F}$ is realized through constraint satisfaction or sheaf-based propagation, \eqref{eq:scenario} inherits compositionality, enabling modular stress-test construction.
\end{remark}

% --------------------------------------------------
% Invariants and stability
% --------------------------------------------------
\section{Invariants and Stability}
\label{sec:invariants}
Key invariants enforced by governance overlays include:
\begin{itemize}
    \item \textbf{Spatial Consistency:} Spatial operations preserve bounded geometric distortion.
    \item \textbf{Temporal Coherence:} Updates respect domain-specific lag bounds.
    \item \textbf{Consensus Gate:} Debate outcomes are enacted only when confidence exceeds the configured threshold.
    \item \textbf{Subject Coverage:} Every active interaction vertex is tagged to a subject in $\mathfrak{S}_t$.
    \item \textbf{Role Alignment:} Role vertices in $\mathfrak{R}_t$ retain expertise edges into $\mathfrak{S}_t$ and instantiated agents.
\end{itemize}

\begin{proposition}[Subject--Role Consistency]
If $\mathcal{I}_{unique}(X_t)$ holds and every interaction vertex is covered by $\mathfrak{S}_t$, then the expertise edges in $\mathfrak{R}_t$ define a well-formed assignment map $\eta_t$ that selects instantiated agents consistent with their subject coverage. Moreover, any violation of expertise alignment must surface as a constraint breach in $\ConstraintsArg{t}$.
\end{proposition}

\begin{proof}[Proof Sketch]
Uniqueness of external identifiers prevents conflicting subject tags. Coverage of interaction vertices ensures each required subject lies in $V^{sub}_t$. The expertise edges $E^{role}_t$ therefore select a unique subject for each role, which by definition of $\eta_t$ corresponds to a deployed agent. Inconsistent assignments would imply a subject without a covering expertise edge or duplicate QID usage, contradicting the assumptions and triggering the constraints bundle.
\end{proof}

\begin{theorem}[Closed-loop Stability]
\label{thm:closed-loop-stability}
Assume $\Phi$ is a contraction and the policy $\Policy$ renders the composite operator $\Graph \mapsto \UpdateOp(\Graph, \Policy(\Graph,I), \Constraints, \Gamma, X, \mathfrak{S}, \mathfrak{R})$ Lipschitz with constant $L < 1$. Then the closed-loop system generated by \eqref{eq:core-update-extended}--\eqref{eq:policy} converges exponentially to a unique fixed point $\Graph^*$.
\end{theorem}

\begin{proof}
We establish exponential convergence through the Banach fixed-point theorem applied to the composite operator $\mathcal{T}(\Graph) := \UpdateOp(\Graph, \Policy(\Graph,I), \Constraints, \Gamma, X, \mathfrak{S}, \mathfrak{R})$.

\textbf{Step 1: Contractivity of $\mathcal{T}$.} 
For any two states $\Graph_1, \Graph_2 \in \mathbb{S}$, we decompose:
\begin{align}
\|\mathcal{T}(\Graph_1) - \mathcal{T}(\Graph_2)\| &= \|\UpdateOp(\Graph_1, \Policy(\Graph_1,I_1), \ldots) - \UpdateOp(\Graph_2, \Policy(\Graph_2,I_2), \ldots)\|
\end{align}

Using the Lipschitz property of $\UpdateOp$ (Assumption~\ref{ass:structured-update-regularity}) and the policy $\Policy$:
\begin{align}
\|\mathcal{T}(\Graph_1) - \mathcal{T}(\Graph_2)\| &\leq L_U \left(\|\Graph_1 - \Graph_2\| + \|\Policy(\Graph_1,I_1) - \Policy(\Graph_2,I_2)\|\right) \\
&\leq L_U(1 + L_P + L_P L_I) \|\Graph_1 - \Graph_2\|
\end{align}

where $L_P$ is the Lipschitz constant of $\Policy$ and $L_I$ accounts for telemetry dependence. By assumption, $L := L_U(1 + L_P + L_P L_I) < 1$.

\textbf{Step 2: Exponential convergence.}
Since $\mathbb{S}$ is complete and $\mathcal{T}$ is a contraction, there exists a unique fixed point $\Graph^*$ with:
\begin{align}
\|\Graph_t - \Graph^*\| \leq L^t \|\Graph_0 - \Graph^*\|
\end{align}
proving exponential convergence with rate $O(L^t)$ where $L < 1$.
\end{proof}

% --------------------------------------------------
% Optimization perspective
% --------------------------------------------------
\section{Optimization Perspective}
Certain policy objectives can be framed as the optimisation problem
\begin{align}
    \begin{aligned}
        \min_{\ActionsArg{t}} \quad & J\bigl(\GraphArg{t+1}, \GraphArg{t}\bigr) \\
        \text{subject to} \quad & \GraphArg{t+1} = \UpdateOp\bigl(\GraphArg{t}, \ActionsArg{t}, \ConstraintsArg{t}, \Govern{t}, X_t, \mathfrak{S}_t, \mathfrak{R}_t\bigr), \\
        & \ActionsArg{t} \in \mathcal{A}_{\text{feasible}}.
    \end{aligned}
    \label{eq:optimisation}
\end{align}
This formulation clarifies trade-offs between convergence speed, resource usage, and compliance satisfaction.

% --------------------------------------------------
% Hook interfaces
% --------------------------------------------------
\section{Hook Interfaces}
Mathematical hooks f16 (CollectData), f17 (ModelUpdate), and f7 (ProposeConsensus) update auxiliary bundles without mutating $\GraphArg{t}$ directly:
\begin{align}
    \Hook_{16}: \quad & (\GraphArg{t}, I_t) \longmapsto (\ActionsArg{t}, D_t), \label{eq:hook16} \\
    \Hook_{17}: \quad & (\GraphArg{t}, D_t) \longmapsto (\ActionsArg{t}, P_t), \label{eq:hook17} \\
    \Hook_{7}: \quad & (\GraphArg{t}, P_t) \longmapsto (\ActionsArg{t}, C_t). \label{eq:hook7}
\end{align}

% --------------------------------------------------
% Extensibility
% --------------------------------------------------
\section{Extensibility via Domain Absorption}
Domain-specific overlays are incorporated through injections $\iota_D: \State \hookrightarrow \State_D$ satisfying the commuting diagram
\begin{equation}
    \iota_D\bigl(\GraphArg{t+1}\bigr) = \UpdateOp_D\bigl(\iota_D(\GraphArg{t}), \ActionsArg{t}, \ConstraintsArg{t}^D, \Govern{t}^D, X_t^D, \mathfrak{S}_t^D, \mathfrak{R}_t^D\bigr).
    \label{eq:domain-absorption}
\end{equation}

\begin{proposition}[Preservation of Proof Obligations]
If \eqref{eq:domain-absorption} holds and the invariants in \cref{sec:invariants} remain intact under $\iota_D$, then stability guarantees obtained in the base domain extend to the absorbed domain.
\end{proposition}

% --------------------------------------------------
% Implementation guidance
% --------------------------------------------------
\section{Implementation Guidance}
\section{Implementation Canon for New Entrants}
For practitioners joining the programme, the following canonical checklist summarises the mandatory mathematical workflow. Each item references the corresponding structures defined earlier in this specification.\n\begin{enumerate}[label=\textbf{Step~\arabic*}]\n    \item \textbf{Initialise structured state:} populate $\State_t$ with $\mathfrak{S}_t$ and $\mathfrak{R}_t$ using federated QIDs (\cref{tab:notation}).\n    \item \textbf{Verify federation uniqueness:} ensure $\mathcal{I}_{unique}(X_t)$ prior to every update (\cref{eq:core-update-extended}).\n    \item \textbf{Execute debate kernel:} apply $\Phi$ when disagreement exceeds thresholds to leverage geometric convergence.\n    \item \textbf{Resolve dating expansion:} compute $\mathcal{L}_t$, evaluate $\psi_{\text{expand}}$, and admit updates via $\mathcal{K}_{\text{dating}}$ (\cref{eq:leaf-set,eq:topology-update}).\n    \item \textbf{Adapt control triggers:} update $\theta_i$ using the PI adaptation law defined for governance thresholds.\n    \item \textbf{Generate scenarios:} compute $\Sigma_t = \mathcal{F}(\GraphArg{t},\ConstraintsArg{t})$ and confirm coverage/minimality.\n    \item \textbf{Enforce invariants:} spatial/temporal/consensus plus $\mathcal{I}_{cover}(\mathfrak{S}_t)$ and $\mathcal{I}_{align}(\mathfrak{R}_t)$.\n    \item \textbf{Apply core evolution:} advance $\GraphArg{t+1} = \UpdateOp(\GraphArg{t},\ActionsArg{t},\ConstraintsArg{t},\Govern{t},X_t,\mathfrak{S}_t,\mathfrak{R}_t)$ and register the integration map $L: \mathcal{M} \leftrightarrow \mathcal{R}$.\n\end{enumerate}\nNumerical implementations of $\UpdateOp$ should maintain stability near equilibrium (e.g., by employing adaptive step control when $
orm{\GraphArg{t+1} - \GraphArg{t}}$ becomes small). Logging hooks ought to persist policy decisions, enabling auditability of the signals emitted by \eqref{eq:policy}.

% --------------------------------------------------
% Conclusion
% --------------------------------------------------
\section{Conclusion}
We have articulated the Enhanced Federated Graph Framework as a rigorously specified discrete-time system with explicit policy, debate, and scenario-generation layers. The modular structure enables analytical proofs, policy experimentation, and domain absorption without perturbing the core evolution law.

% --------------------------------------------------
% References placeholder
% --------------------------------------------------
% \bibliographystyle{plain}
% \bibliography{references}

% --------------------------------------------------
% Implementation guidance
% --------------------------------------------------

% --------------------------------------------------
% APPENDIX: Baseline Specification
% --------------------------------------------------
\appendix
\section{Baseline Core Equation}
\label{sec:baseline}

This appendix presents the \emph{lightweight baseline specification} of the federated graph framework, stripping away the extended bundle components to reveal the essential mathematical structure. This simplified view addresses concerns about complexity while maintaining theoretical rigor.

\subsection{Minimal Core Equation}
The baseline federated graph evolution reduces to:
\begin{equation}
\boxed{\Graph_{t+1} = \Phi(\Graph_t, \mathcal{A}_t, \mathcal{C}_t)}
\label{eq:baseline-core}
\end{equation}

where:
\begin{align}
\Graph_t &\in \mathbb{S} \quad \text{(State space)} \\
\mathcal{A}_t &\in \mathcal{U} \quad \text{(Action space)} \\
\mathcal{C}_t &\in \mathcal{C} \quad \text{(Constraint space)} \\
\Phi &: \mathbb{S} \times \mathcal{U} \times \mathcal{C} \to \mathbb{S} \quad \text{(Update operator)}
\end{align}

This is the \emph{essential mathematical kernel} that captures:
\begin{itemize}
    \item Graph state evolution ($\Graph_t \to \Graph_{t+1}$)
    \item Action-driven updates ($\mathcal{A}_t$ influences transitions)
    \item Constraint satisfaction ($\mathcal{C}_t$ enforces invariants)
\end{itemize}

\subsection{Relationship to Extended Framework}
The full framework from \eqref{eq:core-update-extended} extends this baseline through:
\begin{align}
\text{Extended:} \quad &\Graph_{t+1} = \UpdateOp(\Graph_t, \ActionsArg{t}, \ConstraintsArg{t}, \Govern{t}, X_t, \mathfrak{S}_t, \mathfrak{R}_t) \\
\text{Baseline:} \quad &\Graph_{t+1} = \Phi(\Graph_t, \mathcal{A}_t, \mathcal{C}_t)
\end{align}

The additional components provide:
\begin{align}
\Govern{t} &\to \text{Governance bundles (runtime policy support)} \\
X_t &\to \text{Federation coordination (multi-instance support)} \\
\mathfrak{S}_t &\to \text{Subject matter expertise (domain knowledge)} \\
\mathfrak{R}_t &\to \text{Role instantiation (agent management)}
\end{align}

\subsection{Baseline Theoretical Guarantees}
Even in the simplified baseline form, the framework preserves core guarantees:

\begin{theorem}[Baseline Stability]
\label{thm:baseline-stability}
If $\Phi$ satisfies the Lipschitz condition:
\begin{align}
\|\Phi(\Graph_1, \mathcal{A}, \mathcal{C}) - \Phi(\Graph_2, \mathcal{A}, \mathcal{C})\| \leq L \|\Graph_1 - \Graph_2\|
\end{align}
with $L < 1$, then the baseline system \eqref{eq:baseline-core} converges exponentially to a unique fixed point.
\end{theorem}

\begin{theorem}[Baseline Constraint Preservation]
\label{thm:baseline-constraints}
For any constraint $I \in \mathcal{C}_t$, the baseline update preserves:
\begin{align}
I(\Graph_t) = \text{true} \implies I(\Graph_{t+1}) = \text{true}
\end{align}
\end{theorem}

\subsection{When to Use Each Specification}
\begin{center}
\begin{tabular}{|p{0.4\textwidth}|p{0.25\textwidth}|p{0.25\textwidth}|}
\hline
\textbf{Use Case} & \textbf{Baseline \eqref{eq:baseline-core}} & \textbf{Extended \eqref{eq:core-update-extended}} \\
\hline
Theoretical analysis & ✓ Recommended & Complex but complete \\
Proof-of-concept & ✓ Ideal & Overkill \\
Single-instance deployment & ✓ Sufficient & ✗ Unnecessary \\
Production multi-federation & ✗ Insufficient & ✓ Required \\
Policy-driven governance & ✗ Missing features & ✓ Full support \\
Runtime adaptability & ✗ Limited & ✓ Comprehensive \\
\hline
\end{tabular}
\end{center}

\subsection{Migration Path}
Systems can start with the baseline specification and incrementally adopt extended features:
\begin{enumerate}
    \item \textbf{Phase 1:} Implement baseline $\Phi(\Graph_t, \mathcal{A}_t, \mathcal{C}_t)$
    \item \textbf{Phase 2:} Add governance layer $\Govern{t}$ for policy support
    \item \textbf{Phase 3:} Integrate federation coordination $X_t$ for multi-instance
    \item \textbf{Phase 4:} Deploy subject-matter expertise $\mathfrak{S}_t, \mathfrak{R}_t$ for domain specialization
\end{enumerate}

This appendix demonstrates that the framework's mathematical foundation remains elegant and tractable, even when extended features are required for production deployments.

% --------------------------------------------------
% Adaptive Intelligence Extensions
% --------------------------------------------------
\section{Adaptive Intelligence Extensions}
\label{sec:adaptive-intelligence}

This section extends the core mathematical framework to support adaptive intelligence based on inquiry type classification. The extensions maintain mathematical rigor while enabling context-aware processing optimization.

\subsection{Enriched Formula Specification}

The enriched federated graph formula introduces adaptive intelligence:

\begin{definition}[Adaptive Update Operator]
\label{def:adaptive-update}
The adaptive update operator extends the core framework:
\begin{align}
\Graph_{t+1} &= \text{AdaptiveUpdate}(\Graph_t, \text{InquiryType}(I_t), \mathcal{F}_t^{\text{adaptive}}, \mathcal{A}_t^{\text{specialized}}, d_t^{\text{contextual}}, X_t, \Psi_t, \mathcal{S}_t) \label{eq:adaptive-update}
\end{align}
where:
\begin{align}
\text{InquiryType}(I_t) &\in \{\text{FACT}, \text{ASSIST}, \text{ANALYSIS}, \text{PREDICT}\} \\
\mathcal{F}_t^{\text{adaptive}} &= \text{SelectOverlays}(\mathcal{F}_{1:17}, \text{InquiryType}(I_t), \text{Urgency}(I_t)) \\
\mathcal{A}_t^{\text{specialized}} &= \text{SelectAgents}(\mathcal{A}_t, \text{InquiryType}(I_t)) \\
d_t^{\text{contextual}} &= \text{SelectScaffold}(d_t, \text{InquiryType}(I_t))
\end{align}
\end{definition}

\subsection{Mathematical Branch Decomposition}

\begin{definition}[Specialized Update Functions]
\label{def:specialized-updates}
The adaptive update decomposes into four specialized branches:
\begin{align}
\text{AdaptiveUpdate}(\Graph, I, \mathcal{F}, \mathcal{A}, d, X, \Psi, \mathcal{S}) = \begin{cases}
\text{FactUpdate}(\Graph, \mathcal{F}_{\text{fact}}, \mathcal{A}_{\text{auth}}, d_{\text{verify}}, X, \Psi, \mathcal{S}) & \text{if } I = \text{FACT} \\
\text{AssistUpdate}(\Graph, \mathcal{F}_{\text{impl}}, \mathcal{A}_{\text{prac}}, d_{\text{solve}}, X, \Psi, \mathcal{S}) & \text{if } I = \text{ASSIST} \\
\text{AnalysisUpdate}(\Graph, \mathcal{F}_{\text{comp}}, \mathcal{A}_{\text{anal}}, d_{\text{compare}}, X, \Psi, \mathcal{S}) & \text{if } I = \text{ANALYSIS} \\
\text{PredictUpdate}(\Graph, \mathcal{F}_{\text{trend}}, \mathcal{A}_{\text{fore}}, d_{\text{model}}, X, \Psi, \mathcal{S}) & \text{if } I = \text{PREDICT}
\end{cases}
\end{align}
\end{definition}

\subsection{Contraction Mapping Preservation}

\begin{theorem}[Adaptive Update Contraction]
\label{thm:adaptive-contraction}
If each specialized update function maintains the contraction property:
\begin{align}
\|U_{\text{type}}(\Graph_1) - U_{\text{type}}(\Graph_2)\| \leq \alpha_{\text{type}} \|\Graph_1 - \Graph_2\|
\end{align}
where $\alpha_{\text{type}} < 1$ for each inquiry type, then the adaptive update operator preserves convergence.
\end{theorem}

\begin{proof}[Proof Sketch]
For any inquiry type $I_t$, the adaptive update reduces to the corresponding specialized function. Since each specialized function is contractive with factor $\alpha_{\text{type}} < 1$, convergence is preserved within each branch. The inquiry type classification is deterministic, ensuring consistent branch selection.
\end{proof}

\subsection{Overlay Selection Mathematics}

\begin{definition}[Adaptive Overlay Selection]
\label{def:overlay-selection}
The overlay selection function optimizes computational efficiency:
\begin{align}
\text{SelectOverlays}(\mathcal{F}_{1:17}, I, U) &= \mathcal{F}_{\text{core}}(I) \cup \mathcal{F}_{\text{optional}}(I, U) \\
\text{where}\quad \mathcal{F}_{\text{core}}(I) &= \text{essential overlays for inquiry type } I \\
\mathcal{F}_{\text{optional}}(I, U) &= \text{urgency-dependent optional overlays}
\end{align}
\end{definition}

\begin{lemma}[Overlay Selection Completeness]
\label{lem:overlay-completeness}
For each inquiry type $I$, the selected overlays maintain sufficient coverage:
\begin{align}
\text{Coverage}(\mathcal{F}_{\text{selected}}(I)) \geq \text{Coverage}_{\text{min}}(I)
\end{align}
where $\text{Coverage}_{\text{min}}(I)$ represents the minimum overlay coverage required for convergence in inquiry type $I$.
\end{lemma}

\subsection{Agent Specialization Framework}

\begin{definition}[Agent Category Mapping]
\label{def:agent-categories}
Agents are categorized based on inquiry type requirements:
\begin{align}
\text{Category}: \mathcal{A} &\to \{\text{AUTHORITY}, \text{IMPLEMENTER}, \text{ANALYST}, \text{FORECASTER}\} \\
\text{SelectAgents}(I) &= \{a \in \mathcal{A} : \text{Category}(a) \in \text{Required}(I)\}
\end{align}
where $\text{Required}(I)$ specifies the agent categories needed for inquiry type $I$.
\end{definition}

\subsection{Confidence Calibration Theory}

\begin{definition}[Type-Calibrated Confidence]
\label{def:confidence-calibration}
Confidence scores are calibrated based on inquiry type reliability:
\begin{align}
\text{Conf}_{\text{calibrated}}(r, I, q) &= \min(1, \beta(I) \cdot r \cdot q) \\
\text{where}\quad \beta(I) &= \text{baseline confidence multiplier for inquiry type } I \\
r &= \text{raw confidence score} \\
q &= \text{quality assessment score}
\end{align}
\end{definition}

\begin{proposition}[Confidence Ordering]
\label{prop:confidence-ordering}
The baseline multipliers preserve logical confidence ordering:
\begin{align}
\beta(\text{FACT}) > \beta(\text{ASSIST}) > \beta(\text{ANALYSIS}) > \beta(\text{PREDICT})
\end{align}
reflecting the inherent uncertainty characteristics of each inquiry type.
\end{proposition}

\subsection{Convergence Properties Under Adaptation}

\begin{theorem}[Adaptive Convergence]
\label{thm:adaptive-convergence}
The adaptive framework maintains convergence while optimizing computational efficiency:
\begin{align}
\lim_{t \to \infty} \|\Graph_{t+1} - \Graph_t\| &= 0 \\
\text{subject to}\quad \text{Efficiency}(t) &\geq \text{Efficiency}_{\text{baseline}}
\end{align}
where efficiency is measured by computational resource utilization.
\end{theorem}

\begin{proof}[Proof Strategy]
\begin{enumerate}
    \item Show that overlay selection preserves essential convergence components (Lemma \ref{lem:overlay-completeness})
    \item Prove that specialized agent sets maintain debate convergence through category requirements
    \item Demonstrate that contextual debate structures preserve consensus dynamics
    \item Establish that the branch selection is consistent and deterministic
\end{enumerate}
Complete proofs require empirical validation of efficiency claims and convergence rates.
\end{proof}

\subsection{Urgency-Responsive Complexity}

\begin{definition}[Complexity Adaptation]
\label{def:complexity-adaptation}
Computational complexity adapts to urgency requirements:
\begin{align}
\text{Complexity}(I, U) &= \text{BaseComplexity}(I) \times \text{UrgencyFactor}(U) \\
\text{where}\quad \text{UrgencyFactor}(U) &\in [0.3, 1.0] \text{ based on urgency level}
\end{align}
\end{definition}

\begin{proposition}[Urgency-Quality Tradeoff]
\label{prop:urgency-tradeoff}
The framework maintains acceptable quality bounds under urgency constraints:
\begin{align}
\text{Quality}(I, U) \geq \text{MinAcceptable}(I) \quad \forall U \in \text{Urgency}
\end{align}
\end{proposition}

\subsection{Future Work and Validation Requirements}

The adaptive intelligence extensions require empirical validation in several areas:

\begin{enumerate}
    \item \textbf{Convergence Rate Measurement}: Actual convergence speeds for different inquiry types
    \item \textbf{Efficiency Validation}: Empirical measurement of computational savings
    \item \textbf{Quality Assessment}: User satisfaction and accuracy studies across inquiry types
    \item \textbf{Calibration Tuning}: Optimization of confidence baseline multipliers based on real data
\end{enumerate}

The mathematical framework provides the theoretical foundation for these extensions while maintaining the rigor and provable properties of the original formulation.

\end{document}
